% Options for packages loaded elsewhere
\PassOptionsToPackage{unicode}{hyperref}
\PassOptionsToPackage{hyphens}{url}
\PassOptionsToPackage{dvipsnames,svgnames,x11names}{xcolor}
%
\documentclass[
]{article}

\usepackage{amsmath,amssymb}
\usepackage{iftex}
\ifPDFTeX
  \usepackage[T1]{fontenc}
  \usepackage[utf8]{inputenc}
  \usepackage{textcomp} % provide euro and other symbols
\else % if luatex or xetex
  \usepackage{unicode-math}
  \defaultfontfeatures{Scale=MatchLowercase}
  \defaultfontfeatures[\rmfamily]{Ligatures=TeX,Scale=1}
\fi
\usepackage{lmodern}
\ifPDFTeX\else  
    % xetex/luatex font selection
\fi
% Use upquote if available, for straight quotes in verbatim environments
\IfFileExists{upquote.sty}{\usepackage{upquote}}{}
\IfFileExists{microtype.sty}{% use microtype if available
  \usepackage[]{microtype}
  \UseMicrotypeSet[protrusion]{basicmath} % disable protrusion for tt fonts
}{}
\makeatletter
\@ifundefined{KOMAClassName}{% if non-KOMA class
  \IfFileExists{parskip.sty}{%
    \usepackage{parskip}
  }{% else
    \setlength{\parindent}{0pt}
    \setlength{\parskip}{6pt plus 2pt minus 1pt}}
}{% if KOMA class
  \KOMAoptions{parskip=half}}
\makeatother
\usepackage{xcolor}
\setlength{\emergencystretch}{3em} % prevent overfull lines
\setcounter{secnumdepth}{5}
% Make \paragraph and \subparagraph free-standing
\ifx\paragraph\undefined\else
  \let\oldparagraph\paragraph
  \renewcommand{\paragraph}[1]{\oldparagraph{#1}\mbox{}}
\fi
\ifx\subparagraph\undefined\else
  \let\oldsubparagraph\subparagraph
  \renewcommand{\subparagraph}[1]{\oldsubparagraph{#1}\mbox{}}
\fi


\providecommand{\tightlist}{%
  \setlength{\itemsep}{0pt}\setlength{\parskip}{0pt}}\usepackage{longtable,booktabs,array}
\usepackage{calc} % for calculating minipage widths
% Correct order of tables after \paragraph or \subparagraph
\usepackage{etoolbox}
\makeatletter
\patchcmd\longtable{\par}{\if@noskipsec\mbox{}\fi\par}{}{}
\makeatother
% Allow footnotes in longtable head/foot
\IfFileExists{footnotehyper.sty}{\usepackage{footnotehyper}}{\usepackage{footnote}}
\makesavenoteenv{longtable}
\usepackage{graphicx}
\makeatletter
\def\maxwidth{\ifdim\Gin@nat@width>\linewidth\linewidth\else\Gin@nat@width\fi}
\def\maxheight{\ifdim\Gin@nat@height>\textheight\textheight\else\Gin@nat@height\fi}
\makeatother
% Scale images if necessary, so that they will not overflow the page
% margins by default, and it is still possible to overwrite the defaults
% using explicit options in \includegraphics[width, height, ...]{}
\setkeys{Gin}{width=\maxwidth,height=\maxheight,keepaspectratio}
% Set default figure placement to htbp
\makeatletter
\def\fps@figure{htbp}
\makeatother
\newlength{\cslhangindent}
\setlength{\cslhangindent}{1.5em}
\newlength{\csllabelwidth}
\setlength{\csllabelwidth}{3em}
\newlength{\cslentryspacingunit} % times entry-spacing
\setlength{\cslentryspacingunit}{\parskip}
\newenvironment{CSLReferences}[2] % #1 hanging-ident, #2 entry spacing
 {% don't indent paragraphs
  \setlength{\parindent}{0pt}
  % turn on hanging indent if param 1 is 1
  \ifodd #1
  \let\oldpar\par
  \def\par{\hangindent=\cslhangindent\oldpar}
  \fi
  % set entry spacing
  \setlength{\parskip}{#2\cslentryspacingunit}
 }%
 {}
\usepackage{calc}
\newcommand{\CSLBlock}[1]{#1\hfill\break}
\newcommand{\CSLLeftMargin}[1]{\parbox[t]{\csllabelwidth}{#1}}
\newcommand{\CSLRightInline}[1]{\parbox[t]{\linewidth - \csllabelwidth}{#1}\break}
\newcommand{\CSLIndent}[1]{\hspace{\cslhangindent}#1}

\makeatletter
\makeatother
\makeatletter
\makeatother
\makeatletter
\@ifpackageloaded{caption}{}{\usepackage{caption}}
\AtBeginDocument{%
\ifdefined\contentsname
  \renewcommand*\contentsname{Table of contents}
\else
  \newcommand\contentsname{Table of contents}
\fi
\ifdefined\listfigurename
  \renewcommand*\listfigurename{List of Figures}
\else
  \newcommand\listfigurename{List of Figures}
\fi
\ifdefined\listtablename
  \renewcommand*\listtablename{List of Tables}
\else
  \newcommand\listtablename{List of Tables}
\fi
\ifdefined\figurename
  \renewcommand*\figurename{Figure}
\else
  \newcommand\figurename{Figure}
\fi
\ifdefined\tablename
  \renewcommand*\tablename{Table}
\else
  \newcommand\tablename{Table}
\fi
}
\@ifpackageloaded{float}{}{\usepackage{float}}
\floatstyle{ruled}
\@ifundefined{c@chapter}{\newfloat{codelisting}{h}{lop}}{\newfloat{codelisting}{h}{lop}[chapter]}
\floatname{codelisting}{Listing}
\newcommand*\listoflistings{\listof{codelisting}{List of Listings}}
\makeatother
\makeatletter
\@ifpackageloaded{caption}{}{\usepackage{caption}}
\@ifpackageloaded{subcaption}{}{\usepackage{subcaption}}
\makeatother
\makeatletter
\@ifpackageloaded{tcolorbox}{}{\usepackage[skins,breakable]{tcolorbox}}
\makeatother
\makeatletter
\@ifundefined{shadecolor}{\definecolor{shadecolor}{rgb}{.97, .97, .97}}
\makeatother
\makeatletter
\makeatother
\makeatletter
\makeatother
\ifLuaTeX
  \usepackage{selnolig}  % disable illegal ligatures
\fi
\IfFileExists{bookmark.sty}{\usepackage{bookmark}}{\usepackage{hyperref}}
\IfFileExists{xurl.sty}{\usepackage{xurl}}{} % add URL line breaks if available
\urlstyle{same} % disable monospaced font for URLs
\hypersetup{
  pdftitle={Gross Output Production Functions and Spatial Dependence},
  pdfauthor={Hans Martinez},
  colorlinks=true,
  linkcolor={blue},
  filecolor={Maroon},
  citecolor={Blue},
  urlcolor={Blue},
  pdfcreator={LaTeX via pandoc}}

\title{Gross Output Production Functions and Spatial Dependence}
\author{Hans Martinez}
\date{2023-10-10}

\begin{document}
\maketitle
\ifdefined\Shaded\renewenvironment{Shaded}{\begin{tcolorbox}[sharp corners, interior hidden, boxrule=0pt, enhanced, breakable, frame hidden, borderline west={3pt}{0pt}{shadecolor}]}{\end{tcolorbox}}\fi

\hypertarget{intro}{%
\section*{Intro}\label{intro}}
\addcontentsline{toc}{section}{Intro}

Key Idea: Review latest spatial inference estimators on gross output
production functions framework using firm-level Colombian data (à la T.
Conley, Gonçalves, and Hansen 2018).

Using several dissimilarity measures, I could compare the latest spatial
inference methods, including different ways of clustering the standard
errors. How? I could construct dissimilarity measures using firms'
characteristics such as share of skilled and unskilled labor, or share
of imports and exports. With the dissimilarity measures on hand, I can
also compare MW's (Müller and Watson 2022b, 2022a) and Kernel
estimators. In addition, I can compare traditional s.e.
clustering\footnote{Because these variables are available in the
  Colombian data, it is natural for researchers to cluster the standard
  errors by metropolitan area or by subsectors.} with the
cluster-learning method by Cao et al. (2021).

To the best of my understanding, there is not a guide regarding the
best-practices when dealing with the inference in the estimation of
gross-output production functions. The literature has focused in
obtaining production function and productivity unbiased estimates.

Alternatively, the productivity variance-covariance matrix might be of
interests to researchers and policy-makers. For example, the
productivity variance-covariance matrix across sectors could indicate
how productivity shocks diffuse through other sectors across the
economy. This might be of interest to policymakers because they can
impulse policies that have the largest positive effect on the whole
economy. The Var-Covar matrix within sectors, using skilled/unskilled
labor or local/foreign purchases/sales, might inform researchers about
the productivity spillover effects between firms.

In the gross output production function framework, an output shock and a
productivity shock form the error term. The output shock is independent
of the inputs (usually, capital, labor, and intermediates), is not
serially or cross-sectionally correlated, and has mean zero.

Productivity, on the other hand, is observed by the firm when choosing
input quantities, but unobserved by the econometrician. The well-known
\emph{simultaneity} problem. In addition, productivity is assumed to
follow a Markov process. The rationale is that productivity is
persistent. Researchers assume in practice a linear AR(1) functional
form.

Because production functions are commonly estimated by industry sector,
both the error term and productivity have sector specific variances.
Researchers assume technology in the same industry is common for all
industries. Hence, firms vary only in their productivity.

I can leave the error term as a random output shock and focus on
productivity. If we think about technology and how it diffuses through
the economy and how innovations in one sector spill over to other
sectors and companies, we can back up the estimates with a solid
theoretical model.

\hypertarget{setting}{%
\section{Setting}\label{setting}}

Firms produce gross output \(Y_{it}\) given a production function
\(Y_{it}=G(K_{it},L_{it},M_{it})e^{\omega_{it}+\varepsilon_{it}}\) and a
productivity shock \(\omega_{it}\) using capital \(K_{it}\), labor
\(L_{it}\), and intermediates \(M_{it}\).

\[
y_{it}=g(k_{it},l_{it},m_{it})+\omega_{it}+\varepsilon_{it}
\]

where \(\varepsilon_{it}\) is an output shock that is not part of the
information set of the firm. The productivity shock \(\omega_{it}\) is
known when taking input decisions, giving rise to the well-known
simultaneity problem in estimating production functions. It is usually
assumed that the output shock is iid, and that the productivity shock
follows a Markov process\footnote{In T. G. Conley et al. (2003),
  \(\omega_{it}\) is serially uncorrelated, independent of
  \(\varepsilon_{it}\), expectation zero, but it is correlated across
  sectors as a function of economic distance. \(\varepsilon_{it}\) is
  serially uncorrelated, independent across sectors with sector-specific
  variance.}. In practice, researchers assume an AR(1) model

\[
\omega_{it}=h(\omega_{it-1})+\eta_{it}
\]

where \(\eta_{it}\) is iid

\hypertarget{location-and-dissimilarity-measure}{%
\section{Location and dissimilarity
measure}\label{location-and-dissimilarity-measure}}

The firm-level Colombian data contains information on firm's
characteristics, inputs, and outputs. The data also indicates the firms'
metropolitan area and country region.

I could construct several dissimilarity measures:

\begin{enumerate}
\def\labelenumi{\arabic{enumi}.}
\tightlist
\item
  Labor market (T. G. Conley, Flyer, and Tsiang 2003): The Colombian
  data includes firm-level labor detailed by skilled and unskilled
  labor, as well as managers, local and foreign technicians, their wages
  and benefits.
\item
  Industry sectors (T. G. Conley et al. 2003): In the case of the
  Colombian data, the firm-level survey does not detail sales per
  sector. However, it does include the share of local and export sales,
  and the share of locally and imported inputs. It also includes the
  firm's four-digit sector code; industries are usually defined at the
  three-digit level.
\item
  Firm characteristics: The data also details firms' value of capital in
  land, buildings, machinery, and office equipment; it also contains
  some details on industrial expenses such as maintenance and industrial
  work by other establishment; and general expenses such as publicity
  and insurances.
\end{enumerate}

\hypertarget{to-do}{%
\section{To do}\label{to-do}}

\begin{itemize}
\tightlist
\item
  Investigate if there exist an input-output matrix by industry sector
  for Colombia 80s.
\item
  What are these metropolitan areas? Is there geographical information,
  latitude and altitude?
\item
  Can other dataset be linked to these areas, like a Census.
\end{itemize}

\hypertarget{references}{%
\section*{References}\label{references}}
\addcontentsline{toc}{section}{References}

\hypertarget{refs}{}
\begin{CSLReferences}{1}{0}
\leavevmode\vadjust pre{\hypertarget{ref-Cao2021}{}}%
Cao, Jianfei, Christian Hansen, Damian Kozbur, and Lucciano Villacorta.
2021. {``Inference for Dependent Data with Learned Clusters,''} July.
\url{http://arxiv.org/abs/2107.14677}.

\leavevmode\vadjust pre{\hypertarget{ref-Conley2003}{}}%
Conley, Timothy G, Bill Dupor, Gadi Barlevy, Susanto Basu, Gerald
Carlino, Xiao-Hong Chen, Adeline Delavande, et al. 2003. {``A Spatial
Analysis of Sectoral Complementarity.''} \emph{Journal of Political
Economy}.

\leavevmode\vadjust pre{\hypertarget{ref-Conleyetal2003}{}}%
Conley, Timothy G, Fredrick Flyer, and Grace R Tsiang. 2003.
{``Spillovers from Local Market Human Capital and the Spatial
Distribution of Productivity in Malaysia.''} \emph{Advances in Economic
Analysis and Policy3} 3. \url{http://www.bepress.com/bejeap.}

\leavevmode\vadjust pre{\hypertarget{ref-Conley2018}{}}%
Conley, Timothy, Silvia Gonçalves, and Christian Hansen. 2018.
{``Inference with Dependent Data in Accounting and Finance
Applications.''} \emph{Journal of Accounting Research} 56 (September):
1139--1203. \url{https://doi.org/10.1111/1475-679X.12219}.

\leavevmode\vadjust pre{\hypertarget{ref-Muller2022ECTA}{}}%
Müller, Ulrich K, and Mark W Watson. 2022a. {``SPATIAL CORRELATION
ROBUST INFERENCE.''} \emph{Econometrica} 90: 2901--35.
\url{https://doi.org/10.3982/ECTA19465}.

\leavevmode\vadjust pre{\hypertarget{ref-Muller2022JBES}{}}%
---------. 2022b. {``Spatial Correlation Robust Inference in Linear
Regression and Panel Models.''} \emph{Journal of Business \& Economic
Statistics} 00: 1--15.
\url{https://doi.org/10.1080/07350015.2022.2127737}.

\end{CSLReferences}



\end{document}
